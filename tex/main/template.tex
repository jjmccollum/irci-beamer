\documentclass[11pt]{beamer}
%\documentclass[11pt,aspectration=169]{beamer}% use this instead of the above line if you want your slides to have 16:9 aspect ratio

\title{A Beamer Template for the IRCI}% specify full title in curly braces, short title in brackets
\subtitle{with Examples} % specify subtitle, if any
\date{\today}% specify full event/date in curly braces, short version in brackets
\author{Joey McCollum}% specify author
\institute{Institute for Religion and Critical Inquiry\\Australian Catholic University}% specify author institution

\usetheme{irci}
\usefonttheme{serif} % if you want to use sans serif for body text, comment this out

\begin{document}
	\begin{frame}
		\titlepage
	\end{frame}
	\section{Examples}
	\sectionframe
	\subsection{Lists}
	\begin{frame}
		\frametitle{Slide Title}
		\framesubtitle{Subtitle}
		An itemized list follows:
		\begin{itemize}
			\item Level 1
			\begin{itemize}
				\item Level 2
				\begin{itemize}
					\item Level 3
				\end{itemize}
			\end{itemize}
		\end{itemize}
		\vspace{\baselineskip}

		An enumerated list follows:
		\begin{enumerate}
			\item Level 1
			\begin{enumerate}
				\item Level 2
				\begin{enumerate}
					\item Level 3
				\end{enumerate}
			\end{enumerate}
		\end{enumerate}
	\end{frame}
	\subsection{Quotes}
	\begin{frame}
		Block quotes are also supported:
		\begin{quote}
			How an actual quote should be defined. Blockquote tags should only be used for `quoted' content, because they are presented from paragraphs by screen readers.
		\end{quote}
	\end{frame}
	\subsection{Blocks}
	\begin{frame}
		\LaTeX-style blocks using the ACU palette are also supported:
		\begin{block}{Block}
			A standard block invoked with the \texttt{block} environment.
		\end{block}

		\begin{alertblock}{Alert}
			An alert block invoked with the \texttt{alertblock} environment.
		\end{alertblock}

		\begin{exampleblock}{Example}
			An example block invoked with the \texttt{exampleblock} environment.
		\end{exampleblock}
	\end{frame}
	\subsection{Math}
	\begin{frame}
		Inline math, such as $e^{\pi i} = -1$, is also possible.\par
		\vspace{\baselineskip}

		Likewise, the \texttt{equation} environment works in the usual way:
		\begin{equation}
			P(A|B)=\frac{P(B|A)P(A)}{P(B)}
		\end{equation}
	\end{frame}
\end{document}